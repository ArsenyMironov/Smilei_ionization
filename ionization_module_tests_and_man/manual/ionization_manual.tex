\documentclass[prd, preprint,
aps,
amsmath,
amssymb,
onecolumn,
%twocolumn,
nofootinbib,
%showpacs,
%showkeys,
superscriptaddress,
%longbibliography
]{revtex4-2}
\usepackage{graphicx,dcolumn,bm}
\usepackage{amsfonts,amsmath,amssymb,amsthm,amscd}
\usepackage[english]{babel}
\hyphenation{ALPGEN}
\hyphenation{EVTGEN}
\hyphenation{PYTHIA}
%\usepackage{multirow}
\usepackage{physics}
\sloppy\raggedbottom
\usepackage{needspace}
\usepackage{xcolor}
\usepackage{slashed}
\usepackage[export]{adjustbox}
\usepackage{tabularx}
\usepackage{hyperref}

\begin{document} 
	
	\title{Ionization modules for SMILEI beyond the PPT-ADK formula}
	
	\author{A.~A. Mironov}
	\affiliation{LULI, Sorbonne Université, CNRS, CEA, École Polytechnique, Institut Polytechnique de Paris, F-75255 Paris, France; \url{mironov.hep@gmail.com}}
	
	\maketitle
	
	\section{Ionization rates}
	\subsection{Tunneling ionization (PPT-ADK)}
	All the notations are the same as on the reference page in the SMILEI manual \url{https://smileipic.github.io/Smilei/Understand/ionization.html}:
	\begin{gather}
		\label{PPT_definition}
		\Gamma_{Z^*}=A_{n^*,l^*} B_{l,|m|}\left( \frac{2(2I_p)^{3/2}}{|E|} \right)^{2n^*-|m|-1} \exp\left\lbrace -  \frac{2(2I_p)^{3/2}}{3|E|}\right\rbrace,\\
		A_{n^*,l^*} = \frac{2^{2n^*}}{n^*\Gamma(n^*+l^*+1)\Gamma(n^*-l^*)},\\
		B_{l,|m|} = \frac{(2l+1)\Gamma(l+|m|+1)}{2^{|m|}\Gamma(|m|+1)\Gamma(l-|m|+1)},
	\end{gather}
	where $\Gamma(x)$ is the Gamma function.
	
	\subsubsection{Representation in the code (as in the standard ionization module)}
	
	First, note that the effective orbital angular momentum $l^*=n^*-1$. This allows simplifying $A_{n^*,l^*}$. Second, we assume that the ionization is successive (outer shell electrons are extracted first) and that at the instant of ionization, each electron has the magnetic quantum number $m=0$. The ionization probability for an electron with $m=0$ is higher than for the one with $|m|>0$. The hypothesis is that electrons with $|m|>0$ have enough time to relax to the $m=0$ state once it is not occupied. With this, we set $m=0$ for all the electrons, and rewrite the PPT-ADK formula accordingly (note that in the code notations $Z=Z^*-1$):
	\begin{gather}
		\label{tunnel}
		\Gamma_{Z}=\beta(Z)\exp\left\lbrace -\frac{\Delta}{3} + \alpha(Z)\log\Delta \right\rbrace,\\
		\alpha(Z) = c^*-1,\quad c^*=(Z+1)\sqrt{\frac{2}{I_p}} \equiv 2n^*,\\
		\beta(Z) = I_p\frac{2^{\alpha(Z)} \left[ 8l+4 \right]}{c^*\Gamma(c^*)} ,\\
		\label{gamma_Delta}
		\Delta = \frac{\gamma(Z)}{E}, \quad \gamma(Z)=2(2I_p)^{3/2}. 
	\end{gather}
	
	\subsubsection{Bug fix in the standard model}
	In \texttt{Smilei/src/Ionization/IonizationTunnel.cpp}, in the method \texttt{void IonizationTunnel::operator()}, array \texttt{Dnom[Z]} is not initialized. As a result, for high $Z$ atoms, the array can be uncontrollably allocated with random data, and the result of the operation \texttt{Dnom\_tunnel[k\_times+1] -= D\_sum} is not defined. As the value for \texttt{Dnom[k\_times+1]} is not defined anywhere before this line in the algorithm, the correction \texttt{Dnom\_tunnel[k\_times+1] = -D\_sum} resolves the bug. This correction is implemented in the standard ionization module and all new modules described below. However, it is recommended to review the code and initialize the array properly.
	
	\subsection{Tunnelling ionization (PPT-ADK) with account for $|m|>0$}
	The model adds the dependence on the magnetic quantum number $m$ as in Eq.~\eqref{PPT_definition}. It is implemented in the code by changing the definitions for $\alpha(Z)$ and $\beta(Z)$ in Eq.~\eqref{tunnel} as follows:
	\begin{gather}
		\alpha(Z) = c^*-|m|-1,\\
		\beta(Z) = I_p\frac{2^{\alpha(Z)} \left[ 8l+4 \right]}{c^*\Gamma(c^*)} \times \frac{\Gamma(l+|m|+1)}{\Gamma(|m|+1)\Gamma(l-|m|+1)}.
	\end{gather}
	Note that $\beta(Z)$ depends both on $l^*$ and $l$. 
	
	The model is implemented in file \texttt{Smilei/src/Ionization/IonizationTunnelFullPPT.cpp} and can be used in the namelist by putting \texttt{ionization\_model = 'tunnel\_full\_PPT'} in the corresponding atom \texttt{Species}.
	
	\subsubsection{Implementation details}
	The magnetic quantum number $m$ is attributed to each electron in accordance with the following rules. 
	\begin{enumerate}
		\item Since $\Gamma_Z(m=0)>\Gamma_Z(|m|=1)>\Gamma_Z(|m|=2)>\ldots$, we assume that for electrons with the same azimuthal quantum number $l$, the states with the lowest value of $|m|$ are ionized first.
		
		\clearpage
		\item Electrons with the same azimuthal quantum number $l$ occupy the sub-shells in the order of increasing $|m|$ and for the same $|m|$ in the order of increasing $m$:
		\begin{table}[!h]
			\begin{tabular}{l | c | c | c | c | c | c | c | c | c | c | c}
				Level order num & 1 & 2 &  3 &  4 & 5 & 6 &  7 &  8 & 9 & 10 & $\ldots$ \\ \hline
				Magnetic q. num $m$ & 0 & 0 & -1 & -1 & 1 & 1 & -2 & -2 & 2 & 2 & $\ldots$
			\end{tabular}
		\end{table}
		
		For example, for the atomic configuration with 3 $p$-electrons ($l=1$) in the outer shell,  we choose $m=0,\, 0, -1$ for these electrons. 
		
		
	\end{enumerate}
	With this algorithm, by knowing the atomic number $A$, we can assign a unique set of quantum numbers $nlm$ to each electron on the atomic sub-shells and identify their extraction order during successive ionization. 
	For example, the full configuration for ${}_{12}$Mg and the corresponding order for the ionization process is:
	\begin{table}[h!]
		\begin{tabular}{l | c | c | c | c | c | c | c | c | c | c | c | c}
			Shell & $3s$ & $3s$ & $2p$ & $2p$ & $2p$ & $2p$ & $2p$ & $2p$ & $2s$ & $2s$ & $1s$ & $1s$ \\\hline
			$n$   &    3 &    3 &    2 &    2 &    2 &    2 &    2 &    2 &    2 &    2 &    1 &    1 \\\hline
			$l$   &    0 &    0 &    1 &    1 &    1 &    1 &    1 &    1 &    0 &    0 &    0 &    0 \\\hline
			$m$   &    0 &    0 &    0 &    0 &   -1 &   -1 &    1 &    1 &    0 &    0 &    0 &    0 \\\hline
			Ionization order & 1 & 2 &  3 &  4 & 5 & 6 &  7 &  8 & 9 & 10 & 11 & 12
		\end{tabular}
	\end{table}		

	Note that this hypothesis may be quite strong, as, when the $p$ shell is not full, in principle, electrons could occupy states in different order. Though, it is good enough for the checks whether the effect of $|m|>0$ states is relevant for the simulated physics. 
	
	The values for $m$ for each atom with $A=1\ldots99$ are stored in \texttt{Smilei/src/Ionization/IonizationTables.h} in the table named \texttt{magneticQuantumNumber}, which is similar to the table for $l$ used in the tunnelling formula \eqref{tunnel} in the standard SMILEI package. The table for $m$ was generated based on the table for $l$ following the rules presented above.
	
	\textbf{Important note!} As a temporary solution for the $f$- and higher shells ($l\geq 3$) , $m$ is always set to zero [namely, the rates are calculated with Eqs.~\eqref{tunnel}-\eqref{gamma_Delta}]. This is due to possible sub-level overlap for the $d$- and $f$-shells, that cannot be reproduced with the described algorithm for the table generation. This issue is to be resolved in the future.
	
	
	\clearpage
	\subsection{Tong \& Ling formula}
	The formula proposed by Tong and Lin [\href{https://iopscience.iop.org/article/10.1088/0953-4075/38/15/001}{Tong X. M. and Lin C. D., J. Phys. B: At. Mol. Opt. Phys. 38 2593 (2005)}] extends the tunnelling ionization rate to the barrier-suppression regime. This is achieved by introducing the \textit{empirical} factor in Eq.~\eqref{PPT_definition}:
	\begin{equation}
		\Gamma_{Z^*}^{\text{TL}} = \Gamma_{Z^*}\times \exp\left( -2\alpha_{\text{TL}} n^{*2} \frac{E}{(2 I_p)^{3/2}}\right),
	\end{equation}
	where $\alpha_{\text{TL}}$ is an empirical constant. Typically, the value for it stays in the interval from 6 to 9. The actual value should be guessed from experimental data. When such data is not available, the formula can be used for qualitative analysis of the barrier-suppression ionization (BSI), e.g. see [\href{https://iopscience.iop.org/article/10.1088/1612-202X/ab6559}{M. F. Ciappina and S. V. Popruzhenko., Laser Phys. Lett. 17 025301 (2020)}]. The module was tested to reproduce the results from this paper.
	
	\subsubsection{Representation in the code}
	In the code, the Tong-Ling formula is implemented as follows:
	\begin{gather}
		\label{tonglin}
		\Gamma_{Z}^\text{TL}=\Gamma_{Z}\times \exp\left[ -E\lambda(Z) \right] ,\\
		\lambda(Z) = \frac{\alpha_{\text{TL}} c^{*2}}{\gamma(Z)},
	\end{gather}
	where $\Gamma_Z$ and $\gamma(Z)$ are given in Eqs.~\eqref{tunnel} and \eqref{gamma_Delta}, respectively.
	
	The model is implemented in file \texttt{Smilei/src/Ionization/IonizationTunnelTL.cpp} and can be used in the namelist by putting \texttt{ionization\_model = 'tunnel\_TL'} in the corresponding atom \texttt{Species}. The $\alpha_{\text{TL}}$ can be set with the (newly added) parameter \texttt{ionization\_tl\_parameter = <value>} in the corresponding \texttt{Species}
	(set to 6 by default).
	
	
	\clearpage
	\subsection{BSI formula}
	I. Ouatu implemented a different model of BSI initially proposed in [\href{https://journals.aps.org/pra/abstract/10.1103/PhysRevA.98.043407}{I. Yu. Kostyukov and A. A. Golovanov, Phys. Rev. A 98, 043407 (2018)}]. The original implementation by I. Ouatu is available on Github \url{https://github.com/iouatu/mySmilei}. It was used in the study [\href{https://journals.aps.org/pre/abstract/10.1103/PhysRevE.106.015205}{I. Ouatu et al,
		Phys. Rev. E 106, 015205 (2022)}]. The module is included in this version of SMILEI without changes (except for some small corrections in the header file) and was tested to reproduce the results of this paper.
	
	The model uses the ionization rate Eq.~\eqref{tunnel} when applicable and switches to an empirical formula described in the original paper by Kostyukov \& Golovanov. There are no additional parameters introduced to tune the model. 
	
	The model is implemented in file \texttt{Smilei/src/Ionization/IonizationTunnelBSI.cpp} and can be used in the namelist by putting \texttt{ionization\_model = 'tunnel\_BSI'} in the corresponding atom \texttt{Species}. In the implementation, the code relies on some additional functionality introduced in \texttt{Smilei/src/Ionization/continuity\_tool.cpp}.
	
	
\end{document}